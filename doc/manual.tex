\documentclass[12pt,a4paper]{article}
\usepackage[english, english]{babel} %English 
\usepackage[utf8]{inputenc} %utf8 with more special chars
\usepackage[T1]{fontenc}       % 
\usepackage[a4paper,left=1.5cm,right=1.5cm,top=2.3cm,bottom=2.3cm]{geometry} %Page-geometry

\usepackage{color}			% Colors
\usepackage{amsmath}		% Better Math
\usepackage{graphicx}		% Using Pictures and graphs?
\usepackage{setspace}		% define row distance
\usepackage{amsfonts}		% fonts

\usepackage{ifthen} % We need some basic control

\usepackage{listings} \lstset{ frame= single , breaklines=true} \lstset{language=} 

% Header and footer
\usepackage{fancyhdr}
\pagestyle{fancy}
 \fancyhead[L]{\thepage}
%\fancyhf{}
\renewcommand{\headrulewidth}{0.2pt}
%Footer left
\fancyfoot[L]{kernwaffe.de}
%\fancyfoot[R]{}
%\fancyfoot{}
% Footer right
\fancyhead[R]{Manual for NQ/NOQ}
% Line in footer
%\renewcommand{\footrulewidth}{0.2pt}
\onehalfspacing

%paragraph newline fix
\makeatletter
\renewcommand\paragraph{\@startsection{paragraph}{4}{\z@}%
  {-3.25ex\@plus -1ex \@minus -.2ex}%
  {1.5ex \@plus .2ex}%
  {\normalfont\normalsize\bfseries}}
\makeatother

\newcommand{\notice}[1]{\textcolor{red}{\em (#1)}} % for red notices while working

\newboolean{onKW} %Declaration
\setboolean{onKW}{false} %assignment
%\ifthenelse{\boolean{onKW}}{}{}

\author{ by luborg}  %Autor
\title{An Admins Manual \\
Noquarter and NOQ\\
\includegraphics{noquarter.jpg} \\
\ifthenelse{\boolean{onKW}}{ Kernwaffe Edition }{ Open Edition }
\\
}			%title
\date{ \copyright\today}				%date

\begin{document}

\maketitle

\newpage

\setcounter{page}{2}

\tableofcontents   %TOC

\newpage 
\part{Noquarter}
\section{Builtin Commands}

%This are sections for Kw and open, so we differ
\newcommand{\inclnqcmds}{\subsection{readconfig}
\textbf{Flag:} \hfill G \linebreak
\textbf{Usage:} \hfill !readconfig  \linebreak
\textbf{Description:} \hfill reloads the shrubbot config file and refreshes user flags

\subsection{time}
\textbf{Flag:} \hfill C \linebreak\textbf{Usage:} \hfill !time [name|slot\#] (reason) \linebreak
\textbf{Description:} \hfill displays the local time


\subsection{setlevel}
\textbf{Flag:} \hfill s \linebreak\textbf{Usage:} \hfill !setlevel [name|slot\#] [level) \linebreak
\textbf{Description:} \hfill sets the admin level of a player
 
\subsection{kick}
\textbf{Flag:} \hfill k \linebreak\textbf{Usage:} \hfill !kick (command) \linebreak
\textbf{Description:} \hfill kick a player with an optional reason

\subsection{ban}
\textbf{Flag:} \hfill b \linebreak\textbf{Usage:} \hfill !ban [name|slot\#] (time) (reason) \linebreak
\textbf{Description:} \hfill ban a player by NAME/slot with an optional expiration time (seconds) and reason

\subsection{banguid}
\textbf{Flag:} \hfill b \linebreak\textbf{Usage:} \hfill !banguid [GUID] (time) (reason) \linebreak
\textbf{Description:} \hfill ban a (perhaps not ingame) player by GUID with an optional expiration time (seconds) and reason

\subsection{banip}
\textbf{Flag:} \hfill b \linebreak\textbf{Usage:} \hfill !banip [IP] (time) (reason) \linebreak
\textbf{Description:} \hfill ban a (perhaps not ingame) player by IP with an optional expiration time (seconds) and reason

\subsection{unban}
\textbf{Flag:} \hfill b \linebreak\textbf{Usage:} \hfill !unban [ban slot\#] \linebreak
\textbf{Description:} \hfill unbans a player specified ban number as seen in !showbans

\subsection{put}
\textbf{Flag:} \hfill p \linebreak\textbf{Usage:} \hfill !put [name|slot\#] [r|b|s] \linebreak
\textbf{Description:} \hfill move a player to a specified team

\subsection{pause}
\textbf{Flag:} \hfill Z \linebreak\textbf{Usage:} \hfill !pause  \linebreak
\textbf{Description:} \hfill pauses the game for all players
\linebreak
\textbf{Comment} \hfill will resume game after 180 seconds

\subsection{unpause}
\textbf{Flag:} \hfill Z \linebreak\textbf{Usage:} \hfill !unpause  \linebreak
\textbf{Description:} \hfill unpauses the game if it had been paused by !pause

\subsection{listplayers}
\textbf{Flag:} \hfill i \linebreak\textbf{Usage:} \hfill !listplayers  \linebreak
\textbf{Description:} \hfill display a list of connected players, their slot numbers as well as their admin levels
\linebreak
\textbf{Comment} \hfill get the Punkbuster slotnumber by adding 1 to the normal slotnumber

\subsection{listteams}
\textbf{Flag:} \hfill I \linebreak\textbf{Usage:} \hfill !listteams  \linebreak
\textbf{Description:} \hfill displays info about the teams

\subsection{mute}
\textbf{Flag:} \hfill m \linebreak\textbf{Usage:} \hfill !mute [name|slot\#] \linebreak
\textbf{Description:} \hfill mutes a player
\linebreak
\textbf{Comment} \hfill If no time is entered, the player is muted for g\_defaultMute time. 

\subsection{unmute}
\textbf{Flag:} \hfill m \linebreak\textbf{Usage:} \hfill !unmute [name|slot\#] \linebreak
\textbf{Description:} \hfill unmutes a muted player

\subsection{showbans}
\textbf{Flag:} \hfill B \linebreak\textbf{Usage:} \hfill !showbans (start at ban\#) ((banner) (banner's name)) ((find) (banned player)) ((reason) (reason for ban)) \linebreak
\textbf{Description:} \hfill display a (partial) list of active bans

\subsection{help}
\textbf{Flag:} \hfill h \linebreak\textbf{Usage:} \hfill !help (command) \linebreak
\textbf{Description:} \hfill display commands available to you or help on a specific command

\subsection{admintest}
\textbf{Flag:} \hfill a \linebreak\textbf{Usage:} \hfill !admintest  \linebreak
\textbf{Description:} \hfill display your current admin level

\subsection{cancelvote}
\textbf{Flag:} \hfill c \linebreak\textbf{Usage:} \hfill !cancelvote  \linebreak
\textbf{Description:} \hfill cancels the vote currently taking place

\subsection{passvote}
\textbf{Flag:} \hfill V \linebreak\textbf{Usage:} \hfill !passvote  \linebreak
\textbf{Description:} \hfill passes the vote currently taking place

\subsection{spec999}
\textbf{Flag:} \hfill P \linebreak\textbf{Usage:} \hfill !spec999  \linebreak
\textbf{Description:} \hfill moves 999 pingers to the spectator team
\linebreak
\textbf{Comment} \hfill also moves all players with ping higher as 500 to spec

\subsection{shuffle}
\textbf{Flag:} \hfill S \linebreak\textbf{Usage:} \hfill !shuffle  \linebreak
\textbf{Description:} \hfill shuffle the teams to try and even them

\subsection{rename}
\textbf{Flag:} \hfill N \linebreak\textbf{Usage:} \hfill !rename [name|slot\#] [new name] \linebreak
\textbf{Description:} \hfill renames a player

\subsection{gib}
\textbf{Flag:} \hfill g \linebreak\textbf{Usage:} \hfill !gib (name|slot\#) (reason) \linebreak
\textbf{Description:} \hfill insantly gibs a player

\subsection{slap}
\textbf{Flag:} \hfill A \linebreak\textbf{Usage:} \hfill !slap [name|slot\#] (damage) (reason) \linebreak
\textbf{Description:} \hfill give a player a specified amount of damage for a specified reason
\linebreak
\textbf{Comment} \hfill will never gib a player or kill him

\subsection{burn}
\textbf{Flag:} \hfill U \linebreak\textbf{Usage:} \hfill !burn [name|slot\#] (reason) \linebreak
\textbf{Description:} \hfill burns a player taking some of his health

\subsection{warn}
\textbf{Flag:} \hfill R \linebreak\textbf{Usage:} \hfill !warn [name|slot\#] [reason] \linebreak
\textbf{Description:} \hfill warns a player by displaying the reason

\subsection{news}
\textbf{Flag:} \hfill W \linebreak\textbf{Usage:} \hfill !news (mapname) \linebreak
\textbf{Description:} \hfill play the map's news reel or another map's news reel if specified

\subsection{lock}
\textbf{Flag:} \hfill K \linebreak\textbf{Usage:} \hfill !lock [r|b|s|all] \linebreak
\textbf{Description:} \hfill lock one or all of the teams from players joining

\subsection{unlock}
\textbf{Flag:} \hfill K \linebreak\textbf{Usage:} \hfill !unlock [r|b|s|all] \linebreak
\textbf{Description:} \hfill unlock one or all locked teams

\subsection{nade}
\textbf{Flag:} \hfill x \linebreak\textbf{Usage:} \hfill !nade [name|slot\#] \linebreak
\textbf{Description:} \hfill makes a player drop a bunch of grenades or all players

\subsection{pip}
\textbf{Flag:} \hfill z \linebreak\textbf{Usage:} \hfill !pip [name|slot\#] \linebreak
\textbf{Description:} \hfill show sparks around a player or all players

\subsection{pop}
\textbf{Flag:} \hfill z \linebreak\textbf{Usage:} \hfill !pop [name|slot\#] \linebreak
\textbf{Description:} \hfill pops the helmets of a player or all players

\subsection{restart}
\textbf{Flag:} \hfill r \linebreak\textbf{Usage:} \hfill !restart  \linebreak
\textbf{Description:} \hfill restarts the current map

\subsection{reset}
\textbf{Flag:} \hfill r \linebreak\textbf{Usage:} \hfill !reset  \linebreak
\textbf{Description:} \hfill resets the current match

\subsection{fling}
\textbf{Flag:} \hfill L \linebreak\textbf{Usage:} \hfill !fling [name|slot\#] \linebreak
\textbf{Description:} \hfill flings a player in a random direction

\subsection{throw}
\textbf{Flag:} \hfill L \linebreak\textbf{Usage:} \hfill !throw [name|slot\#] \linebreak
\textbf{Description:} \hfill throws a player forward

\subsection{launch}
\textbf{Flag:} \hfill L \linebreak\textbf{Usage:} \hfill !launch [name|slot\#] \linebreak
\textbf{Description:} \hfill launch a player vertically

\subsection{disorient}
\textbf{Flag:} \hfill d \linebreak\textbf{Usage:} \hfill !disorient [name|slot\#] (reason) \linebreak
\textbf{Description:} \hfill flips a player's view

\subsection{orient}
\textbf{Flag:} \hfill d \linebreak\textbf{Usage:} \hfill !orient [name|slot\#] \linebreak
\textbf{Description:} \hfill unflips a disoriented player's view

\subsection{resetxp}
\textbf{Flag:} \hfill X \linebreak\textbf{Usage:} \hfill !resetxp [name|slot\#] (reason) \linebreak
\textbf{Description:} \hfill reset a player's XP

\subsection{resetmyxp}
\textbf{Flag:} \hfill M \linebreak\textbf{Usage:} \hfill !resetmyxp  \linebreak
\textbf{Description:} \hfill resets your own XP

\subsection{nextmap}
\textbf{Flag:} \hfill n \linebreak\textbf{Usage:} \hfill !nextmap  \linebreak
\textbf{Description:} \hfill loads the next map

\subsection{swap}
\textbf{Flag:} \hfill w \linebreak\textbf{Usage:} \hfill !swap  \linebreak
\textbf{Description:} \hfill swap teams

\subsection{revive}
\textbf{Flag:} \hfill v \linebreak\textbf{Usage:} \hfill !revive [name|slot\#] (reason) \linebreak
\textbf{Description:} \hfill revives a dead player
\linebreak
\textbf{Comment} \hfill doesn't work on gibbed players

\subsection{rocket}
\textbf{Flag:} \hfill j \linebreak\textbf{Usage:} \hfill !rocket  \linebreak
\textbf{Description:} \hfill have a rocket shoot from the player who uses this command

\subsection{disguise}
\textbf{Flag:} \hfill T \linebreak\textbf{Usage:} \hfill !disguise [name|slot\#] (class number) \linebreak
\textbf{Description:} \hfill gives a player a disguise of the specified class

\subsection{poison}
\textbf{Flag:} \hfill U \linebreak\textbf{Usage:} \hfill !poison (name|slot\#) (reason) \linebreak
\textbf{Description:} \hfill poisons a player

\subsection{ammopack}
\textbf{Flag:} \hfill J \linebreak\textbf{Usage:} \hfill !ammopack [name|slot\#] \linebreak
\textbf{Description:} \hfill an ammo pack will spawn out of the player if they have enough chargebar

\subsection{medpack}
\textbf{Flag:} \hfill J \linebreak\textbf{Usage:} \hfill !medpack [name|slot\#] \linebreak
\textbf{Description:} \hfill a health pack will spawn out of the player if they have enough chargebar

\subsection{pants}
\textbf{Flag:} \hfill t \linebreak\textbf{Usage:} \hfill !pants (name|slot\#) \linebreak
\textbf{Description:} \hfill removes a player's pants!

\subsection{give}
\textbf{Flag:} \hfill e \linebreak\textbf{Usage:} \hfill !give [name|slot\#] item/thing \linebreak
\textbf{Description:} \hfill gives a player something...
\linebreak
\textbf{Comment} \hfill if you specify a negativ ammount, you can also take (ammo/health)

\subsection{dw}
\textbf{Flag:} \hfill D \linebreak\textbf{Usage:} \hfill !dw (name|slot\#) \linebreak
\textbf{Description:} \hfill drops a player's primary and secondary weapons!
\linebreak
\textbf{Comment} \hfill best for cheaters

\subsection{finger}
\textbf{Flag:} \hfill f \linebreak\textbf{Usage:} \hfill !finger [name|slot\#] \linebreak
\textbf{Description:} \hfill gives specific information about a player

\subsection{uptime}
\textbf{Flag:} \hfill u \linebreak\textbf{Usage:} \hfill !uptime  \linebreak
\textbf{Description:} \hfill displays server uptime

\subsection{glow}
\textbf{Flag:} \hfill o \linebreak\textbf{Usage:} \hfill !glow (name|slot\#) \linebreak
\textbf{Description:} \hfill makes player(s) glow

\subsection{freeze}
\textbf{Flag:} \hfill E \linebreak\textbf{Usage:} \hfill !freeze (name|slot\#) \linebreak
\textbf{Description:} \hfill freezes player(s) move
\linebreak
\textbf{Comment} \hfill also he will not take damage while frozen

\subsection{unfreeze}
\textbf{Flag:} \hfill E \linebreak\textbf{Usage:} \hfill !unfreeze (name|slot\#) \linebreak
\textbf{Description:} \hfill makes player(s) moving again


 }
\newcommand{\inclnqkwcmds}{\include{nqcmdskw} }

\ifthenelse{\boolean{onKW}}{\inclnqcmds}{\inclnqkwcmds}

\newpage

\part{NOQ}
\section{Featurelist}
NOQ Features include, but are not limited to:

\subsection{Database Connectivity}

The NOQ has the possibility to use a DMBS as a backend for most of its advanced commands.
This is strongly encouraged, as most features like bans, mutes and XPSave rely on this.
\\
The configpart for the DBMS:
\\
\\
\begin{tabular}{l l|l}
\hline & & \\
&	["dbms"]="Type", 			& Possible: mySQL and SQLite \\
&	["dbname"]="dbname", 		& Databasename, if Sqlite then name of DBFile\\
&	["dbuser"]="dbuser", 		& Only needed for mySQL\\
&	["dbpassword"]="dbpass", 	& Only needed for mySQL\\
&	["dbhostname"]="dbhost", 	& Only needed for mySQL\\
&	["dbport"]="3306",    		& Only needed for mySQL\\
 & &  \\
 \hline
\end{tabular}	
		

\subsection{Bans/Mutes}
If you have a DBMS activated, Mutes and Bans are automaticly active.
Banning and muting is only possible trough a custom command or direct DB-Interaction till now.

\subsection{XPSave}
Also active as soon as a DBMS is accessible.
The config has some options:
\\
\\
\begin{tabular}{l l|l}
\hline & & \\
& 	["recordbots"] = "0", 	& This will toggle if bots are recorded into the DB. Values:(0/1)\\
& 	["xprestore"] = "1",	& This will enable XPRestore out of database, wich works best on multiple Servers\\
 & &  \\
 \hline
\end{tabular}

\newpage

\subsection{Custom Commands}

The NOQ's most useful feature are its custom commands, wich are oriented at the ETAdminMod syntax, but are more powerful. They are defined in noq\_commands.cfg
\\
The basic Syntax is:

  $[level] - commandname = [command]$
\\
mind the whitespace before the level!
Comments can be done via
$\#comment$
\\
All players with equal or higher level will be able to use the command. \\
If you want ingame help for a command, you have to add a helptext(and optional the syntax):
\begin{lstlisting}
[level] - commandname = [command]
help	= this is the helptext
syntax	= here goes the syntax
[level] - nextcommandname = [nextcommand]
\end{lstlisting}
Config options for Commands are:
\begin{table}[h]
\begin{tabular}{l l|l}
\hline & & \\
 &	 ["usecommands"] = "0", & General switch to disable/enable the commands. Values:(0/1)\\
 & ["commandprefix"] = "!", & the prefix used to trigger the command. \\ 
 & &  \\
 \hline
\end{tabular}
\end{table}
\\
Like in ETAdminmod, you can use several placeholders, wich will be replaced by their value:
\begin{table}[h]\footnotesize
\begin{tabular}{l l|l}
\hline & & \\
 & $<$PARAMETER$>$					& text followed by the command.  \\
 && \begin{small}
(Can be used to enter multiple values.)
\end{small} \\
 & $<$CLIENT\_ID$>$					& the client id of the calling player.\\
 & $<$PLAYER\_CLASS$>$ 				& class of calling player\\
 & $<$PLAYER\_TEAM$>$ 				& side / team of calling player \\
 & $<$PLAYER$>$						& Name of the calling player (without color codes)\\
 & $<$COLOR\_PLAYER$>$				& Name of the calling player (with color codes)\\
 & $<$GUID$>$						& Guid of the calling player\\
 
 & $<$PLAYER\_ LAST\_VICTIM\_ID$>$ 		& 		\\
 & $<$PLAYER\_ LAST\_VICTIM\_NAME$>$ 	&		\\
 & $<$PLAYER\_ LAST\_VICTIM\_CNAME$>$ 	&		\\
 & $<$PLAYER\_ LAST\_VICTIM\_WEAPON$>$ 	&		\\

 & $<$PLAYER\_ LAST\_KILLER\_ID$>$ 		&		\\ 
 & $<$PLAYER\_ LAST\_KILLER\_NAME$>$ 	& 		\\ 
 & $<$PLAYER\_ LAST\_KILLER\_CNAME$>$ 	& 		\\ 
 & $<$PLAYER\_ LAST\_KILLER\_WEAPON)$>$ & 		\\ 
 
 & $<$PART2\_CLASS$>$					& class				\\
 & $<$PART2\_TEAM$>$ 					& team				\\
 & $<$PART2CNAME$>$						& colored name		\\
 & $<$PART2ID$>$						& id				\\
 & $<$PART2PBID$>$						& punkbuster slotnumber\\
 & $<$PART2GUID$>$						& guid				\\
 & $<$PART2LEVEL$>$						& adminlevel		\\
 & $<$PART2NAME$>$						& name without color\\

 & &  \\
 \hline
\end{tabular}
\end{table}



\paragraph{Serverconsole Commands}
This is the basic type of command, wich is just a rework of Noquarters own custom commands.
Both are directly executed at the serverconsole.
Examples:

\begin{lstlisting}
 2 - swap                    	= swap_teams
 0 - beer			= qsay A nice sexy waitress brings ^7<COLOR_PLAYER>^7 a nice cup of beer!
\end{lstlisting}



\paragraph{Shell Commands} 
Commands starting with \$SHL\$ are shellcommands, and are executed in an OS-Shell.
Especially handy on Linux.
The output of the command is sent to the ingame chat.

\begin{lstlisting}
 2 - showserverinfo                 = $SHL$ /usr/local/bin/show.pl <COLOR_PLAYER>
\end{lstlisting}



\paragraph{Lua Commands}

Commands starting with \$LUA\$ will be parsed and executed as Lua-code in the LuaVM, thus able to access al of NOQ's and ET's informations and data structures.

\begin{lstlisting}
 0 - showmaps = $LUA$ showmaps()
 2 - evener = $LUA$ checkBalance(true) # this calls the Evener and allows him to take action
 0 - cmdlist = $LUA$ listCMDs(<CLIENT_ID>, "<PARAMETER>"); # !cmdlist command
  0 - ratio = $LUA$ et.trap_SendConsoleCommand(et.EXEC_APPEND, "chat \" ^2Kills^7\\^1TKS ^2" .. et.gentity_get(<CLIENT_ID>, "sess.kills") .."^7\\^1".. et.gentity_get(<CLIENT_ID>, "sess.team_kills") .. "^7" ); 
 # Yeah, thats a bigger one. It will show Kill/Death ratio
\end{lstlisting}

You may notice that in most examples just one function is called.
Some functions are predefined in the mod and therefore these builtin commands can be easy renamed and unlocked for different levels.


\subsection{Greetings}

The NOQ can greet players individually, and he can announce publicly that a specific player just joined the game.
Public announcements per level are saved in noq\_greetings.cfg (see for it for an example and customize it), while the personal message is in the main config:
\\
\begin{tabular}{l l|l}
\hline & & \\
 &	["persgamestartmessage"] = "Welcome", & the message wich is displayed for the joining player.	\\
 & & \begin{tiny}
please note that the playername is added at the end of the string, separated by a comma.
\end{tiny} \\
 &	["persgamestartmessagelocation"] = "cpm",			& the location where the message will be printed	\\ 
 & &  \\
 \hline
\end{tabular}

\subsection{Pussyfactor}

For every kill, we add a value to the clients number, and to determine the the Pussyfactor, we divide that number trough the number of his kills multiplicated with 100.
If we add 100 for an MP40/Thompsonkill, if the player does only those kills , he will stay at pussyfactor 1.
If we add more or less(as 100) to the number, his pf will rise or decline.
 \\
Pussyfactor $<$ 1 		means he made "cool kills" = poison, goomba, knive
Pussyfactor = 1 		means he makes normal kills
Pussyfactor $>$ 1      means he does uncool kills (Panzerfaust, teamkills, arty?)
\\
As we add 100 for every normal kill, the pussyfactor approaches 1 after some time with "normal" kills.
\\
Config:\\ 
\begin{tabular}{l l|l}
\hline & & \\
 & ["pussyfactor"] = "1", & enable/disable the pussyfactor Values:(0/1) \\
  & &  \\
 \hline
\end{tabular}


\subsection{The Evener}

NOQ's main mechanism to improve gameplay is the Evener. 
Basically it checks all X seconds if teams are unfair, and then notifies or takes action.
Stage one is a warning combined with a request to all players to even the teams.
This will happen if the teams uneven, wich means 3 or more players difference.
After 3 warnings action will be taken and a random player from bigger team will be moved to the smaller team.
If the difference between teams is greater or equal to 5, teams will be shuffled after 1 warning.
Config options are:
\\
\begin{tabular}{l l|l}
\hline & & \\
 & 	["evenerCheckallSec"] = "XX", & time in seconds between each evenercheck. 40 - 80 is recommended  \\
  & &  \\
 \hline
\end{tabular}

\subsection{Selfkill restriction}
You can restrict selfkills on your server to a specific number.
\\
\begin{tabular}{l l|l}
\hline & & \\
 & 	["maxSelfKills"] = "X", & X is the number of selfkills you want to allow. -1 to disable  \\
  & &  \\
 \hline
\end{tabular}


\subsection{Poll restriction}
Basic restrictions for votes are also avaiable, but a better system is in development.
Till now, you have the option to define a min-distance between votes and restrict the nextmapvote to a specific time. Both restrictions don't work if the player has set the shrubbot flag "7".
\\
\begin{tabular}{l l|l}
\hline & & \\
 & 	["polldistance"] = "XXX",  & time in seconds between votes, -1 to disable \\
 &	["nextmapVoteSec"] = "0", &  seconds after/before mapstart/end in wich nextmap vote is allowed.\\
 & & 0 to disable \\
  & &  \\
 \hline
\end{tabular}

\subsection{Offlinemessages and register command}
Offlinemessages are extended private messages wich are provided by NOQ.
To use offlinemessages players need to register on the server via the clientconsole:
\\
\begin{lstlisting}
]/register
Syntax for the register Command: /register username password
Username is your desired username (for web & offlinemessages)
Password will be your password for your webaccess

]/register name passwort
Successfully registered. To reset password just re-register.
\end{lstlisting}
After registering the player can check if he has messages:
\\
\begin{lstlisting}
]/mail
No new offlinemessages
\end{lstlisting}

To write a offlinemessage use the new clientcommand "om"(analog to pm)
\begin{lstlisting}
]/om
Check your syntax: '/command receiver message'.

]/om name test
 Following message was sent to 'name(lastusedname)'
 'test'
\end{lstlisting}
"lastusedname" is the name last used by the receiving player(wich also needs to be registered).
As we just sent a message to player 'name' (wich are weself) we now should have mesages:
\\
\begin{lstlisting}
]/mail

*** NEW OFFLINEMESSAGES ***
*** MESSAGE 1***
*** From: name MSGID: XXXX ***
*** Message: test ***
\end{lstlisting}
Even if the receiving player is offline or on another server, he will receive the message.
To erase use "rmom":
\\
\begin{lstlisting}
]/rmom XXXX
Erased MessageID XXX
\end{lstlisting}
Now your inbox should be empty.

If you join a server and hear the PM-messaging sound, you should check your inbox for mail.


\subsection{New Servercommands}
The NOQ also provides new Servercommands:

\subsubsection{csay}
\textbf{Usage:} \hfill csay [slot\#] text \linebreak
\textbf{Description:} \linebreak
csay will print a text to the console of a player:
\begin{lstlisting}
csay 0 "this is a test"
\end{lstlisting}
will print "this is a test" in the clientconsole of the player in slot 0.

\subsubsection{plock}
\textbf{Usage:} \hfill plock [slot\#] team(r/b/s) time \linebreak
\textbf{Description:} \linebreak
"plock" is the player analogon to the !lock command for teams, except it is the other way round:
The player is locked to a specific team, and won't be able to leave it.
Time is the time in seconds, after wich the player is able to leave the team again.

\section{Commands}

%This are sections for KW-use-only, not in svn.
\newcommand{\inclkwcmds}{\include{kwcmds} }
\newcommand{\inclkwadmins}{\include{kwadmins} }

\ifthenelse{\boolean{onKW}}{\inclkwcmds}{}
\ifthenelse{\boolean{onKW}}{\inclkwadmins}{}


\end{document}
